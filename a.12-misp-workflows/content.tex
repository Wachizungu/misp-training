% DO NOT COMPILE THIS FILE DIRECTLY!
% This is included by the other .tex files.

\begin{frame}[t,plain]
\titlepage
\end{frame}

\begin{frame}
    \frametitle{Content of the presentation}
    \begin{itemize}
        \item MISP Workflows fundamentals
        \item Getting started
        \item Design of the system \& how it can be extended
    \end{itemize}

    \begin{center}
        \frame{\includegraphics[width=0.9\linewidth]{pictures/overview.png}}
    \end{center}
\end{frame}

\begin{frame}
    \frametitle{What problems are we trying to tackle}
    \begin{itemize}
        \item Initial idea came during GeekWeek7.5\footnote{\href{https://cyber.gc.ca/en/events/geekweek-75}{Workshop organized by the Canadian Cyber Center}} \includegraphics[width=0.3\linewidth]{pictures/geekweek75.jpg}
        \item Needs:
        \begin{itemize}
            \item Prevent default MISP behaviors
            \item Hook specific actions to run callbacks
        \end{itemize}
        \item Use-cases:
        \begin{itemize}
            \item Prevent publication of events not meeting some criterias
            \item Prevent querying thrid-party services (e.g. virustotal) with sensitive information
            \item Send notifications in a chat rooms
            \item And much much more..
        \end{itemize}
    \end{itemize}
\end{frame}

\section{Workflow - Fundamentals}
\begin{frame}
    \frametitle{Simplistic overview of a Workflow in action}
    \begin{enumerate}
        \item An \textbf{action} happens in MISP
        \item If there is an \textbf{enabled} Workflow for that \textbf{action}, run it
        \item If all went fine, MISP \textbf{continue} to perform the action
        \begin{itemize}
            \item The operation can potentially be cancelled by \texttt{blocking} modules
        \end{itemize}
    \end{enumerate}
\end{frame}

\begin{frame}
    \frametitle{Terminology}
    \begin{itemize}
        \item \textbf{workflow}: Sequence of all operations (nodes) to be executed. Basically the whole graph.
        \item \textbf{execution path}: A path composed of nodes
        \item \textbf{trigger}: Starting point of a workflow. Triggers are called when specific actions happen in MISP
        \begin{itemize}
            \item A trigger can only have one workflow and vice-versa
        \end{itemize}
    \end{itemize}
    \begin{center}
        \frame{\includegraphics[width=1.0\linewidth]{pictures/simple-workflow.png}}
    \end{center}
\end{frame}

\begin{frame}
    \frametitle{Workflow execution process}
    Typical execution process:
    \begin{enumerate}
        \item An action happens in MISP
        \item The workflow associated to the trigger is ran
        \item Execution result?
        \begin{itemize}
            \item \texttt{\color{green!50!black}success}: Continue the action
            \item \texttt{\color{red}failure} | \texttt{\color{blue}blocked}: Cancel the action
        \end{itemize}
    \end{enumerate}
    \vspace{0.5em}
    Example for Event publish:
    \begin{enumerate}
        \item An Event is about to be published
        \item MISP executes the workflow listening to the \texttt{event-publish} trigger
        \begin{itemize}
            \item {\bf\color{green!50!black}success}: Continue the publishing action
            \item {\bf\color{red}failure} | \texttt{\color{blue}blocked}: Stop publishing and log the reason
        \end{itemize}
    \end{enumerate}
\end{frame}

\begin{frame}
    \frametitle{Blocking and non-blocking Workflows}
    Currently 2 types of workflows:
    \vspace{0.5em}
    \begin{itemize}
        \item {\bf Blocking}: Completion of the action can be prevented
        \begin{itemize}
            \item If a \textbf{blocking module} blocks the action
            \item If a \textbf{blocking module} raises an exception
        \end{itemize}
        \vspace{0.5em}
        \item {\bf Non-blocking}: Workflow execution outcome has no impact
        \begin{itemize}
            \item \textbf{Blocking modules} can still stop the execution
        \end{itemize}
    \end{itemize}
\end{frame}

\begin{frame}
    \frametitle{Execution context}
    \begin{itemize}
        \item Workflows can be triggered by \textbf{any users}
        \item Workflows can be triggered by actions done via the \textbf{UI} or \textbf{API}
        \item However, the user for which the workflow executes has:
        \begin{itemize}
            \item The \texttt{site-admin} permission
            \item Is from the \texttt{MISP.host\_org\_id}
        \end{itemize}
        \item Ensures data is processed regardless of ownership and access: \textbf{no ACL}
    \end{itemize}
\end{frame}

\begin{frame}
    \frametitle{Classes of Workflow modules}
    \begin{center}
        \includegraphics[width=0.6\linewidth]{pictures/module-type.png}
    \end{center}
    3 classes of modules
    \begin{itemize}
        \item \textbf{action}: Allow to executes functions, callbacks or scripts
        \begin{itemize}
            \item Can stop execution
            \item e.g. Webhook, block the execution, perform enrichments, ...
        \end{itemize}
        \item \textbf{logic}: Allow to redirect the execution flow.
        \begin{itemize}
            \item IF condition, fork the blocking execution into a non-blocking one, ...
        \end{itemize}
        \item \textbf{blueprint}: Allow to reuse composition of modules
        \begin{itemize}
            \item Can save subworkflows and its module's configuration
        \end{itemize}
    \end{itemize}
\end{frame}

\begin{frame}
    \frametitle{Sources of Workflow modules}
    3 sources of action modules
    \begin{itemize}
        \item Built-in \textbf{default} modules
        \begin{itemize}
            \item Part of the MISP codebase
            \item \texttt{\scriptsize \textbf{app/Model/}WorkflowModules/action/[module\_name].php}
        \end{itemize}
        \item User-defined \textbf{custom} modules
        \begin{itemize}
            \item Written in PHP
            \item Can extend existing default modules
            \item Can use MISP's built-in functionalities (restsearch, enrichment, push to zmq, ...)
            \item Faster and easier to implement new complex behaviors
            \item \texttt{\scriptsize \textbf{app/Lib/}WorkflowModules/action/[module\_name].php}
        \end{itemize}
    \end{itemize}
\end{frame}

\begin{frame}
    \frametitle{Sources of Workflow modules}
    3 sources of action modules
    \begin{itemize}
        \item Modules from the \textbf{enrichment service}
        \begin{itemize}
            \item \textbf{Default} and \textbf{custom} modules
            \item From the \textit{misp-module} \includegraphics[width=0.25\linewidth]{pictures/misp-module-icon.png}
            \item Written in Python
            \item Can use any python libraries
            \item New \textit{misp-module} module type: \texttt{action}
        \end{itemize}
    \end{itemize}
    \vspace{1em}
    \begin{center}
        $\rightarrow$ Both the PHP and Python systems are \textbf{plug-and-play}
    \end{center}
\end{frame}

\begin{frame}
    \frametitle{Triggers currently available}
    Currently 8 triggers can be hooked. 3 being \textbf{blocking}.
    \includegraphics[width=1.0\linewidth]{pictures/triggers.png}
\end{frame}

\section{Workflow - Getting started}
\begin{frame}
    \frametitle{Getting started with workflows (1)}
    Review MISP settings:
    \begin{enumerate}
        \item Make sure \texttt{MISP.background\_jobs} is turned on
        \item Make sure workers are up-and-running and healthy
        \item Turn the setting \texttt{Plugin.Workflow\_enable} on
        \begin{center}
            \includegraphics[width=0.70\linewidth]{pictures/settings-1.png}
        \end{center}
        \item {[optional:misp-module]} Turn the setting \texttt{Plugin.Action\_services\_enable} on
        \begin{center}
            \includegraphics[width=0.70\linewidth]{pictures/settings-2.png}
        \end{center}
    \end{enumerate}
\end{frame}

\begin{frame}[fragile]
    \frametitle{Getting started with workflows (2)}
    If you wish to use action modules from \texttt{misp-module}, make sure to have:
    \begin{itemize}
        \item The latest update of \texttt{misp-module}
        \begin{itemize}
            \item There should be an \texttt{action\_mod} module type in \url{misp-modules/misp\_modules/modules}
        \end{itemize}
        \item Restarted your \texttt{misp-module} application
    \end{itemize}
    \vspace{1em}
    \begin{lstlisting}[language=text,firstnumber=1]
# This command should show all `action` modules
$ curl -s http://127.0.0.1:6666/modules | \
jq '.[] | select(.meta."module-type"[] | contains("action")) | 
{name: .name, version: .meta.version}'
    \end{lstlisting}
\end{frame}

\begin{frame}
    \frametitle{Getting started with workflows (3)}
    \begin{enumerate}
        \item Go to the list of modules
        \begin{itemize}
            \item \texttt{Administration > Workflows > List Modules}
            \item or \url{/workflows/moduleIndex}
        \end{itemize}
        \item Make sure \textbf{default} modules are loaded
        \item {[optional:misp-module]} Make sure \textbf{misp-module} modules are loaded
    \end{enumerate}
\end{frame}

\begin{frame}
    \frametitle{Creating a workflow with the editor}
    \begin{enumerate}
        \item Go to the list of triggers \texttt{Administration > Workflows}
        \item Enable and edit a \texttt{trigger} from the list
        \item Drag an \texttt{action} module from the side panel to the canvas
        \item From the \texttt{trigger} output, drag an arrow into the \texttt{action}'s input (left side)
        \item Execute the action that would run the trigger and observe the effect!
    \end{enumerate}
    \begin{center}
        \frame{\includegraphics[width=0.7\linewidth]{pictures/triggers.png}}
    \end{center}
    \begin{center}
        \frame{\includegraphics[width=0.50\linewidth]{pictures/editor-1.png}}
    \end{center}
\end{frame}

\begin{frame}
    \frametitle{Working with the editor}
    Operations not allowed:
    \begin{itemize}
        \item Execution loop are not authorized
        \begin{itemize}
            \item Current caveat: If an action re-run the workflow in any way
        \end{itemize}
    \end{itemize}
    \begin{center}
        \frame{\includegraphics[width=0.7\linewidth]{pictures/editor-not-allowed-1.png}}
    \end{center}
\end{frame}

\begin{frame}
    \frametitle{Working with the editor}
    Operations not allowed:
    \begin{itemize}
        \item Multiple connections from the same output
        \begin{itemize}
            \item Execution order not guaranted and confusing for users
        \end{itemize}
    \end{itemize}
    \begin{center}
        \frame{\includegraphics[width=0.7\linewidth]{pictures/editor-not-allowed-2.png}}
    \end{center}
\end{frame}

\begin{frame}
    \frametitle{Working with the editor}
    Operations showing a warning:
    \begin{itemize}
        \item \textbf{Blocking} modules after a \textbf{concurrent tasks} module
        \item \textbf{Blocking} modules in a \textbf{non-blocking} workflow
    \end{itemize}
    \begin{center}
        \frame{\includegraphics[width=1.0\linewidth]{pictures/editor-warning-1.png}}
    \end{center}
\end{frame}

\begin{frame}
    \frametitle{Workflow blueprints}
    \begin{enumerate}
        \item Blueprints allow to \textbf{re-use parts} of a workflow in another one
        \item Blueprints can be saved, exported and \textbf{shared}
    \end{enumerate}
    \begin{center}
        \includegraphics[width=0.5\linewidth]{pictures/blueprint-debugging.png}
    \end{center}
    Blueprints origins:
    \begin{enumerate}
        \item From the "official" \texttt{misp-workflow-blueprints} repository
        \item Created or imported by users
    \end{enumerate}
\end{frame}

\begin{frame}
    \frametitle{Workflow blueprints: Create}
    Select one or more modules to be saved as blueprint then click on the \texttt{save blueprint} button
    \begin{center}
        \includegraphics[width=0.85\linewidth]{pictures/blueprint-1.png}
    \end{center}
\end{frame}

\begin{frame}[fragile]
    \frametitle{Hash path filtering}
    \begin{itemize}
        \item Some modules have the possibility to filter or check conditions using CakePHP's path expression.
    \end{itemize}
\begin{lstlisting}[language=javascript,firstnumber=1]
$path_expression = '{n}[name=fred].id';
$users = [
    {'id': 123, 'name': 'fred', 'surname': 'bloggs'},
    {'id': 245, 'name': 'fred', 'surname': 'smith'},
    {'id': 356, 'name': 'joe', 'surname': 'smith'},
];
$ids = Hash::extract($users, $path_expression);
// => $ids will be [123, 245]
\end{lstlisting}
\begin{center}
    \includegraphics[width=0.4\linewidth]{pictures/module-if-generic.png}
\end{center}
\end{frame}

\begin{frame}
    \frametitle{Module filtering}
    \begin{itemize}
        \item Some action modules accept \textbf{filtering} conditions
        \item E.g. the \texttt{enrich-event} module will only perform the enrichment on Attributes having a \texttt{tlp:white} Tag
    \end{itemize}
    \begin{center}
        \includegraphics[width=0.7\linewidth]{pictures/module-filtering.png}
    \end{center}
\end{frame}

\begin{frame}
    \frametitle{Data format in Workflows}
    \begin{center}
        \includegraphics[width=0.7\linewidth]{pictures/workflow-trigger.png}
    \end{center}
    \begin{itemize}
        \item All triggers will inject data in a workflow
        \item In some cases, there is no format (e.g. User after-save)
        \item In others, the format is \textbf{compliant with the MISP Core format}
        \item In addition to the RFC, the passed data has \textbf{additional properties}
        \begin{itemize}
            \item Attributes are \textbf{always encapsulated} in the Event or Object
            \item Additional key \textbf{\texttt{\_AttributeFlattened}}
            \item Additional key \textbf{\texttt{\_allTags}}
            \item Additional key \textbf{\texttt{inherited}} for Tags
        \end{itemize}
    \end{itemize}
\end{frame}

\begin{frame}
    \frametitle{Logic module: Concurrent Task}
    \begin{itemize}
        \item Special type of \textbf{logic} module allowing multiple connections
        \item Allows \textbf{breaking the execution} flow into a concurrent tasks to be executed later on by a background worker
        \item As a side effect, blocking modules \textbf{cannot cancel} ongoing operations
    \end{itemize}
    \begin{center}
        \frame{\includegraphics[width=0.45\linewidth]{pictures/module-concurrent.png}}
    \end{center}
\end{frame}

\begin{frame}
    \frametitle{Debugging Workflows: Log Entries}
    \begin{itemize}
        \item Workflow execution is logged in the application logs:
        \begin{itemize}
            \item \texttt{/admin/logs/index}
        \end{itemize}
        \item Or stored on disk in the following file:
        \begin{itemize}
            \item \texttt{/app/tmp/logs/workflow-execution.log}
        \end{itemize}
        \item Use the \texttt{webhook-listener.py} tool
        \begin{itemize}
            \item \texttt{/tools/misp-workflows/webhook-listener.py}
        \end{itemize}
    \end{itemize}
    \begin{center}
        \includegraphics[width=1.0\linewidth]{pictures/workflow-debug.png}
    \end{center}
\end{frame}

\begin{frame}
    \frametitle{Debugging Workflows: Debug mode}
    \begin{itemize}
        \item The \includegraphics[width=70px]{pictures/debug-mode.png} can be turned on for each workflows
        \item Each nodes will send data to the provided URL
        \begin{itemize}
            \item Configure the setting: \texttt{Plugin.Workflow\_debug\_url}
        \end{itemize}
        \item Result can be visualized in
        \begin{itemize}
            \item \textbf{offline}: \texttt{tools/misp-workflows/webhook-listener.py}
            \item \textbf{online}: \url{requestbin.com} or similar websites
        \end{itemize}
    \end{itemize}
    \begin{center}
        \includegraphics[width=0.6\linewidth]{pictures/request-bin.png}
    \end{center}
\end{frame}

\section{Learning by examples}
\begin{frame}
    \frametitle{Workflow example 1}
    \begin{center}
        \frame{\includegraphics[width=1.0\linewidth]{pictures/example-1a.png}}
    \end{center}

    \begin{enumerate}
        \item The \texttt{Event-Publish} trigger uses the MISP core format
        \item The \texttt{IF::Tag} module checks if at least one of the Attribute has the \texttt{tlp:white} tag
        \item If it does, the \texttt{Push-to-ZMQ} module will be executed
    \end{enumerate}
\end{frame}

\begin{frame}
    \frametitle{Workflow example 2}
    \begin{center}
        \includegraphics[width=1.0\linewidth]{pictures/example-2a.png}
    \end{center}

    \begin{itemize}
        \item If an event has the \texttt{tlp:red} tag or any of the attribute has it, the publish process will be cancelled
    \end{itemize}
\end{frame}


\section{Extending the system}
\begin{frame}
    \frametitle{Creating a new module in PHP}
    \begin{center}
        \includegraphics[width=0.65\linewidth]{pictures/custom-1.png}
    \end{center}

    \begin{itemize}
        \item \texttt{\small \textbf{app/Lib/}WorkflowModules/action/[module\_name].php}
        \item Module configuration are defined as public variables
        \item The \texttt{exec} function has to be implemented.
        \begin{itemize}
            \item If it returns \textbf{true}, execution will proceed
            \item If it returns \textbf{false}
            \begin{itemize}
                \item And the module is blocking, the execution will stop and the operation will be blocked
            \end{itemize}
        \end{itemize}
    \end{itemize}
\end{frame}


\begin{frame}
    \frametitle{Creating a new module in Python}
    \begin{center}
        \includegraphics[width=0.65\linewidth]{pictures/custom-2.png}
    \end{center}

    \begin{itemize}
        \item Module configuration are defined in the \texttt{moduleinfo} and \texttt{moduleconfig} variables
        \item The \texttt{handler} function has to be implemented.
        \item Blocking logic is the same as other modules
    \end{itemize}
\end{frame}

